% resume-sample.tex 23 Oct 89
% An example using the resume style option
% By Stephen Gildea

% original version 28 Sep 88
% minor changes 23 Oct 89

\documentstyle[resume]{article}
\begin{document}
{\sf
\name{{\large \bf Yunghsiang S.\@ Han}\\
{\small Graduate Institute of Communication Engineering\\
National Taipei University\\151. University Rd., Sanshia, Taipei County\\
Taiwan, R. O. C.\\ yshan@mail.ntpu.edu.tw}} } \hrule

\begin{llist}

\sectiontitle{Education}
\employer{Syracuse University} \location{Syracuse, NY}
\vspace{-.2in}
\begin{itemize}
\item {Ph.D. in Computer and Information Science \location{(August 1993)}
\vspace{-0.2in}
   \begin{itemize}
   \item{{\em Dissertation Topic: Efficient Soft-Decision Decoding
Algorithms \\
for
Linear Block Codes Using Algorithm A*.}}
   \item{{\em\bf Winner of Syracuse University Doctoral Prize of
   the Year 1994.}}
   \end{itemize}}
\end{itemize}

\employer{National Tsing Hua University } \location{Hsinchu,
Taiwan, R. O. C.}
\vspace{-.2in}
\begin{itemize}
\item {MS in Electrical Engineering \location{(June 1986)}}
\item {BS in Electrical Engineering \location{(June 1984)}}
\end{itemize}

\sectiontitle{Professional \\Experience}
\employer{ Graduate Institute of Communication Engineering\\
National Taipei University} \location{Taiwan, R. O. C.}
\dates{August 2004 -- present}
Professor and Chairperson.
\employer{ Department of Computer Science and Information Engineering\\
National Chi Nan University}
\location{Taiwan, R. O. C.}
\dates{August 1998 -- July 2004}
Professor. \employer{The New York State Center for Advanced
Technology in\\ Computer Applications and Software Engineering
(CASE)\\
The Center for Systems Assurance (CSA)\\
Department of
Electrical Engineering and Computer Science\\
 Syracuse University}\location{Syracuse NY, USA}
\dates{September 2002 -- January 2004}
SUPRIA (Syracuse University Prototypical Research in Information
Assurance)\\ Visiting Research Scholar.
\employer{ Department of Electrical Engineering\\
University of Hawaii at Manoa} \location{Honolulu HI , USA}
\dates{June 2001 -- October 2001}
Visiting Scholar.
\employer{ Department of Computer Science and Information Engineering\\
National Chi Nan University} \location{Taiwan, R. O. C.}
\dates{August 1998 -- July 2001}
The Head of Computer and Network Center.
\employer{ Department of Computer Science and Information Engineering\\
National Chi Nan University}
\location{Taiwan, R. O. C.}
\dates{August 1997 -- July  1998}
Associate Professor.
\employer{ Department of Electronic Engineering\\
Hua Fan College of Humanities and Technology} \location{Taiwan, R.
O. C.}
\dates{September 1994 -- July 1996}
The Head of Computer Center.
\employer{ Department of Electronic Engineering\\
Hua Fan College of Humanities and Technology}
\location{Taiwan, R. O. C.}
\dates{August 1993 -- July 1997}
Associate Professor.


\employer{ Department of  Computer and Information Science \\
Syracuse University} \location{Syracuse, NY}
\dates{August 1992 -- August 1993}
Graduate Research Associate.

\employer{ Department of  Computer and Information Science \\
Syracuse University} \location{Syracuse, NY}
\dates{August 1989 -- July 1992}
Graduate Teaching Assistant.

\sectiontitle{Professional\\Services}

\employer{IEEE Information Theory Society Taipei Chapter}\location{Taiwan, R.
O. C.}
\dates{August 2005--
July 2007} Chapter Chair.

\employer{The First IEEE International Conference on Wireless and\\
 Mobile
Computing, Networking and Communications}\location{ Montreal,
Canada}\dates{2005} Technical Program Committee member.

\employer{The Second IEEE International Conference on Wireless and\\ Mobile
Computing, Networking and Communications}\location{ Montreal,
Canada}\dates{2006} Technical Program Committee member.

\employer{The IEEE International Conference on Sensor\\ Networks, Ubiquitous,
and Trustworthy Computing}\location{Taichung, Taiwan, R. O. C.}\dates{2006}
Technical Program Committee member.


\employer{The IEEE International Workshop on Ad Hoc,\\ Ubiquitous
Computing}\location{Taichung, Taiwan, R. O. C.}\dates{2006} Co-Chair.

\employer{International Conference on Algorithms, Systems, and\\ Applications
of Wireless Network}\location{ Xian, China}\dates{2006} Technical Program
Committee member.

\employer{The IEEE Consumer Comminations and Networking\\ Conference
 2007: Wireless Networking Track}\location{Las
Vegas, USA}\dates{2007} Technical Program Committee member.

\employer{The 2007 IFIP International Conference on Embedded and\\ Ubiquitous
Computing }\location{Taipei, Taiwan, R. O. C.}\dates{2007} Technical Program
Committee member.

\employer{The Third IEEE International Conference on Wireless and\\ Mobile
Computing, Networking and Communications}\location{New York City,
USA}\dates{2007} Technical Program Committee member.

\employer{IEEE Global Telecommunications Conference\\ (IEEE GLOBECOM 2007):
Communications Software\\ and Services Symposium}\location{Washington, D.C.,
USA}\dates{2007} Technical Program Committee member.

\employer{The First IEEE Conference on Wireless Rural and Emergency
Communications (WRECOM 2007)}\location{Rome, Italy}\dates{2007} Technical
Program Committee member.

\bigskip
%\newpage
\sectiontitle{Awards \& \\Honors }
\begin{itemize}
\item{2002-2004, {\em SUPRIA Visiting Research Scholarship} --
Awarded by CASE center at Syracuse University, New York.}
 \item
{2000,  {\em 89-fiscal-year Research Award} --
 Awarded by National Science Council, Taiwan, ROC.}
\item {2000, {\em Research Award}  --
 Awarded by National Chi Nan University, Taiwan, ROC.}
\item {1999,  {\em 88-fiscal-year Research Award} --
 Awarded by National Science Council, Taiwan, ROC.}
\item {1998, {\em 87-fiscal-year Research Award} --
 Awarded by National Science Council, Taiwan, ROC.}
\item {1997, {\em 86-fiscal-year Research Award} --
 Awarded by National Science Council, Taiwan, ROC.}
\item { {\bf 1997},  A paper was honored as long presentation at {\it
the 1997 IEEE International Symposium on Information theory.}
\footnote{Only
   papers with the potential to have an impact on the state of the art of their respective research areas
    are accepted as long presentations. Usually there are only 17 of 580 accepted papers to be honored
    as long presentations.}}

\item{{\bf 1994}, {\em Syracuse University Doctoral Prize of the Year 1994} --
Awarded by Syracuse University.}
\item {1994, {\em 83-fiscal-year Research Award} --
 Awarded by National Science Council, Taiwan, ROC.}
\item {1993, {\em 82-fiscal-year Research Award for Young
Researcher}
-- Awarded by National Science Council, Taiwan, ROC.}

\item { {\bf 1993}, A paper was honored as long presentation at {\it
the 1993 IEEE International Symposium on Information theory.}}

\end{itemize}

\bigskip

\sectiontitle{Professional \\Memberships}
\begin{itemize}
\item{Member of IEEE -- Information Theory and Communication Societies.}
%\item{Member of SIAM.}
\end{itemize}

\bigskip

\sectiontitle{Research \\Interests}
\begin{itemize}
\item{Wireless Networks-- especially on the security, energy
control, and analysis of sensor networks and ad hoc networks.}

\item{Security-- especially on the topics related to sensor
networks and privacy-preserving.}

\item{Coding Theory-- especially on the development of the theory
of decoding and the design of practical decoding algorithms for
error-correcting codes.}

\item{Wireless Communication-- especially on the application of
error-correcting codes.}

\item{Interconnection Network-- especially on the distance problem
on an interconnection network.} \item{Algorithms-- especially on
the application of information theoretic concepts for design
algorithms.}


\end{itemize}


\sectiontitle{Publications}

\begin{itemize}
\item {\bf\large Book Chapters}
\begin{enumerate}
\item \underline{Y. S. Han} and P.-N. Chen, ``Sequential Decoding of  Convolutional Codes,''
 {\it Encyclopedia of Telecommunications} (Editor: John Proakis), New York, Wiley,
 2002, pp. 2140-2146.
\end{enumerate}
\item {\bf\large Refereed papers}
\begin{enumerate}

\item H.-T. Pai and \underline{Y. S. Han}, ``Power-Efficient Direct-Voting Assurance for Data Fusion
in Wireless Sensor Networks,'' {\it IEEE Trans. on Computers}, to appear. {\bf
(full paper)}

\item J. Deng, \underline{Y. S. Han}, P.-N. Chen, and P. K. Varshney, ``Optimal Transmission Range for
Wireless Ad Hoc Networks Based on Energy Efficiency,'' {\it IEEE Trans. on
Communications}, to appear. {\bf (full paper)}


\item S.-L. Shieh, P.-N. Chen, and \underline{Y. S. Han}, ``Flip CRC Modification
for Message Length Detection,'' {\it IEEE Trans. on Communications}, to appear.
{\bf (full paper)}



\item H.-T. Pai, J.-T. Sung, and \underline{Y. S. Han}, ``Adaptive
Retransmission with Balanced Load for Fault-Tolerant Distributed Detection in
Wireless Sensor Networks,'' {\it Journal of Information Science and
Engineering: special issue on Wireless Ad Hoc and Sensor Networks}, to appear.
{\bf (full paper)}

\item C. Yao, P.-N. Chen, T.-Y. Wang, \underline{Y. S. Han}, and P. K.
Varshney, ``Performance Analysis and Code Design for Minimum Hamming Distance
Fusion in Wireless Sensor Networks,'' {\it IEEE Trans. on Information Theory},
pp. 1706-1715, May, 2007. {\bf (full paper)}

\item Y.-J. Chen, D.-R. Duh, and \underline{Y. S. Han}, ``An Improved Modulo
$(2^n+1)$ Multiplier for IDEA,'' {\it Journal of Information Science and
Engineering},  pp. 911-923, March 2007.

\item C.-W. Chang, P.-N. Chen, and \underline{Y. S. Han}, ``A Systematic Bit-wise
Decomposition of M-ary Symbol Metric,'' {\it IEEE Trans. on Wireless
Communications}, pp. 2742-2751, October, 2006. {\bf (full paper)}

\item \underline{Y. S. Han}, J. Deng,  and Z. J. Haas, ``Analyzing
Multi-Channel Medium Access Control Schemes with ALOHA Reservation,"
{\it IEEE Trans. on Wireless Communications},  pp. 2143-2152,
August, 2006. {\bf (full paper)}

\item T.-Y. Wang, \underline{Y. S. Han}, B. Chen, and P. K.
Varshney, ``A Combined Decision Fusion and Channel Coding Scheme for
Distributed Fault-Tolerant Classification in Wireless Sensor
Networks,'' {\it IEEE Trans. on Wireless Communications}, pp.
1695-1705, July, 2006. {\bf (full paper)}



\item J. Deng, \underline{Y. S. Han},  and Z. J. Haas, ``Analyzing
Split Channel Medium Access Control Schemes," {\it IEEE Trans. on
Wireless Communications},  pp. 967-971, May, 2006.





\item W. Du, J. Deng, \underline{Y. S. Han}, and P. K.
Varshney ''A Key Pre-distribution Scheme for Sensor Networks Using
Deployment Knowledge,'' {\it IEEE Trans. on Dependable and Secure
Computing},  pp. 62-77, January, 2006. {\bf (full paper)}

\item J. Deng, \underline{Y. S. Han}, W. B. Heinzelman, and P. K.
Varshney, ``Scheduling Sleeping Nodes in High Density Cluster-based
Sensor Networks,'' {\it ACM/Kluwer MONET Special Issue on ``Energy
Constraints and Lifetime Performance in Wireless Sensor Networks,''}
pp. 825-835, December, 2005. {\bf (full paper)}

\item T.-Y. Wang, \underline{Y. S. Han}, and P. K. Varshney,
``Fault-Tolerant Distributed Classification Based on Non-binary
Codes in Wireless Sensor Networks," {\it IEEE Communications
Letters}, pp. 808-810, September, 2005.


\item J. Deng, \underline{Y. S. Han}, W. B. Heinzelman, and P. K.
Varshney, ``Balanced-energy Sleep Scheduling Scheme for High
Density Cluster-based Sensor Networks," {\it Computer
Communications : special issue on ASWN04}, pp. 1631-1642,
September, 2005. {\bf (full paper)}

 \item W. Du, J. Deng, \underline{Y. S. Han}, P. K.
Varshney, J. Katz, and A. Khalili, ``A Pairwise Key
Pre-distribution Scheme for Wireless Sensor Networks," {\it  ACM
Trans. on Information and System Security (TISSEC)}, pp. 228-258,
May, 2005. {\bf (full paper)}




\item T.-Y. Wang, \underline{Y. S. Han}, P. K. Varshney, and P.-N.
Chen, ``Distributed Fault-Tolerant Classification in Wireless
Sensor Networks,'' {\it IEEE Journal on Selected Areas in
Communications (JSAC): special issue on Self-Organizing
Distributed Collaborative Sensor Networks}, pp. 724-734, April,
2005. {\bf (full paper)}



\item C.-C. Lee, P.-C. Chung, D.-R. Duh, \underline{Y. S. Han},
and C.-W. Lin, ``A Practice of a Collaborative Multipoint Medical
Teleconsultation System on Broadband Network,'' {\it Journal of
High Speed Networks}, pp. 207-222, September, 2004. {\bf (full
paper)}

\item \underline{Y. S. Han}, P.-N. Chen and H.-B. Wu, ``A
Maximum-Likelihood Soft-Decision Sequential Decoding Algorithm for
Binary Convolutional Codes,'' {\it IEEE Trans. on Communications},
pp. 173-178, February, 2002.
\item P.-N. Chen and \underline{Y. S. Han}, ``Asymptotic
Minimum Covering Radius of Block Codes,'' {\it SIAM Journal on
Discrete Mathematics}, pp. 549-564, November, 2001. {\bf (full
paper)}


\item P.-N. Chen, T.-Y. Lee, and \underline{Y. S. Han}, ``Distance-Spectrum
Formulas on the Largest Minimum Distance of Block Codes, '' {\it
IEEE Trans. on Information Theory}, pp. 869-885, May, 2000. {\bf
(full paper)}

\item  \underline{Y.  S.  Han},
 ``A New Decoding Algorithm for Complete Decoding of Linear Block Codes,'' {\it SIAM Journal on
Discrete Mathematics}, pp. 664-671, November, 1998. {\bf (full paper)}


\item  \underline{Y.  S.  Han},
 ``A New Treatment of Priority-First Search Maximum-Likelihood
Soft-Decision Decoding of Linear Block Codes,'' {\it IEEE Trans. on
Information Theory}, pp. 3091-3096,  November, 1998.

\item \underline{Y.  S.  Han}, C. R. P. Hartmann, and K. G. Mehrotra, ``Decoding
 Linear Block Codes Using a Priority-First Search: Performance
 Analysis and Suboptimal Version,''{\it IEEE Trans. on
Information Theory}, pp. 1233-1246, May, 1998.

\item \underline{Y.  S.  Han}, and C.  R.  P.  Hartmann, ``The Zero-Guards Algorithm
for General Minimum Distance Decoding Problem," {\it IEEE Trans. on
Information Theory}, pp. 1655-1658, September, 1997.

\item D. L. Tao, C. R. P. Hartmann , and \underline{Y.  S.  Han},
``New Encoding/Decoding Methods for Designing Fault-Tolerant Matrix
Operations, "{\it IEEE Trans. on Parallel and Distributed Systems},
pp. 931-938, September, 1996. {\bf (full paper)}


\item \underline{Y.  S.  Han}, C.  R.  P.  Hartmann, and C-C.  Chen, ``Efficient
Priority-First Search Maximum-Likelihood Soft-Decision Decoding of Linear
Block Codes," {\it IEEE Trans.
on Information Theory}, pp. 1514-1523, September, 1993. {\bf (full paper)}
\end{enumerate}



\item {\bf\large Refereed Conference}

\begin{enumerate}

\item  S.-L.~Shieh, P.-N Chen and \underline{Y. S.
Han},``Reduction of Computational Complexity and Sufficient Stack Size of the
MLSDA by Early Elimination,'' {\it the IEEE International Symposium on
Information Theory (ISIT2007)}, Nice, France, June, 2007.

\item P.-N. Chen, T.-Y. Wang, \underline{Y. S. Han}, and Y.-T. Wang, ``On the
Design of Soft-Decision Fusion Rule for Coding Approach in Wireless
Sensor Networks,'' {\it International Conference on Algorithms,
Systems, and Applications (WASA2006)}, Xian, P. R. China, August,
2006. {\it Lecture Notes in Computer Science (LNCS)},
Springer-Verlag, pp. 140-150, 2006.

\item P.-N. Chen, T.-Y. Wang, \underline{Y. S. Han}, P. K. Varshney, C. Yao,
and S.-L. Shieh, ``Fault-Tolerance Analysis of a Wireless Sensor
Network with Distributed Classification Codes,'' {\it the IEEE
International Symposium on Information Theory (ISIT2006)}, Seattle,
July, 2006.


\item  H.-T. Pai and \underline{Y. S. Han}, ``Power-Efficient Data Fusion
Assurance Using Direct Voting Mechanism in Wireless Sensor
Networks,'' {\it the 2006 IEEE International Conference on Sensor
Networks, Ubiquitous, and Trustworthy Computing (SUTC2006)},
Taichung, Taiwan, June, 2006.

\item  H.-T. Pai, J.-T. Sung, and \underline{Y. S. Han}, ``Adaptive
Retransmission for Distributed Detection in Wireless Sensor
Networks,'' {\it the IEEE Workshop on Ad Hoc and Ubiquitous
Computing (AHUC2006)}, Taichung, Taiwan, June, 2006.

\item J. Deng and \underline{Y. S. Han}, ``Using MDS Codes for
the Key Establishment of Wireless Sensor Networks,'' {\it
International Conference on Mobile Ad-hoc and Sensor Networks (MSN
'05)}, Wuhan, P. R. China, December  2005. {\it Lecture Notes in
Computer Science (LNCS)}, Springer-Verlag, pp. 732-744, 2005.


\item  S.-L.~Shieh, S.-T.~Kuo, P.-N Chen and \underline{Y. S.
Han}, ``Strategies for Blind Transport Format Detection Using Cyclic
Redundancy Check in UMTS WCDMA,'' {\it 2005 IEEE International
Conference on Wireless and Mobile Computing, Networking and
Communications (WIMOB2005)}, Montreal, Canada, August, 2005, pp.
44-50.

\item H.-T. Pai, J.-T. Sung, and \underline{Y.  S. Han}, ``A
Simple Two-Dimensional Coded Detection Scheme in Wireless Sensor
Networks," {\it the First IEEE International Workshop in
Heterogeneous Wireless Sensor Networks (HWISE-2005) }, Fukuoka,
Japan, July 2005, pp. 383-387.


\item P.-N. Chen, T.-Y. Wang, \underline{Y.  S. Han}, P. K.
Varshney and C. Yao, ``Asymptotic Performance Analysis for
minimum-Hamming-distance fusion," {\it the IEEE International
Conference on Acoustics, Speech, and Signal Processing 2005
(ICASSP'05)}, Philadelphia, USA, March 2005, pp. 865-868.

\item S.-L. Shieh, P.-N. Chen, and \underline{Y. S. Han},``A Novel
Modification of Cyclic Redundancy Check for Message Length
Detection," {\it the 2004 IEEE International Symposium on
Information Theory and its Applications (ISITA2004)}, Parma,
Italy, October, 2004.

\item C.-W. Chang, P.-N. Chen, and \underline{Y. S. Han},
``Realization of a Systematic Bit-wise Decomposition Metric," {\it
the 2004 IEEE Asia-Pacific Conference on Circuits and Systems
(APCCAS'04)}, Tainan, Taiwan, December, 2004, pp. 1065-1068.

\item J. Deng, \underline{Y. S. Han}, W. B. Heinzelman, and P. K.
Varshney, ``Balanced-energy Sleep Scheduling Scheme for High
Density Cluster-based Sensor Networks," {\it 4th Workshop on
Applications and Services in Wireless Networks (ASWN04)}, Boston,
Massachusetts, August, 2004. {\bf(Selected for possible
publication in a special issue of Elsevier's Computer
Communications Journal)}

\item Y.-J. Chen, D.-R. Duh, and \underline{Y. S. Han}, ``A New
Modulo $(2^n+1)$ Multiplier for IDEA," {\it the 2004 International
Conference on Security and Management (SAM'04)}, Las Vegas,
Nevada, June, 2004, pp. 318-324.


\item  T.-Y. Wang, \underline{Y. S. Han}, and P. K. Varshney, ``A
Combined Decision Fusion and Channel Coding Scheme for
Fault-Tolerant Classification in Wireless Sensor Networks," {\it
the 2004 IEEE International Conference on Acoustics, Speech, and
Signal Processing (ICASSP 2004)}, Montreal, Quebec, Canada, May,
2004, pp. 1073-1076.

\item J. Deng, \underline{Y. S. Han}, P.-N. Chen, and P. K.
Varshney, ``Optimum Transmission Range for Wireless Ad Hoc
Networks," {\it the IEEE Wireless Communications and Networking
Conference 2004 (WCNC04)},  Atlanta, GA, March, 2004, pp.
1024-1029.

\item W. Du, \underline{Y. S. Han}, and S. Chen
''Privacy-Preserving Multivariate Statistical Analysis: Linear
Regression and Classification,'' {\it the 2004 SIAM International
Conference on Data Mining (SDM04)}, Lake Buena Vista, FL, April,
2004, pp. 222-233. (Regular paper)
 \item T.-Y. Wang, \underline{Y. S.
Han}, and P. K. Varshney, ``Further Results on Fault-Tolerant
Distributed Classification Using Error Correcting Codes," {\it the
SPIE's Aerosense conference on Multisensor, Multisource
Information Fusion: Architectures, Algorithms, and Applications},
Orlando, FL, April, 2004.

\item W. Du, J. Deng, \underline{Y. S. Han}, S. Chen and P. K.
Varshney ''A Key Management Scheme for Wireless Sensor Networks
Using Deployment Knowledge,'' the {\it IEEE INFOCOM 2004}, Hong
Kong, March 2004, pp.586-597.

 \item W. Du, J. Deng, \underline{Y. S. Han}, and P. K.
Varshney, ``A Pairwise Key Pre-distribution Scheme for Wireless
Sensor Networks," {\it Proceedings of 10th ACM Conference on
Computer and Communications
 Security (CCS2003)}, Washington DC,  October, 2003, pp. 42-51.
\item J. Deng, \underline{Y. S. Han}, and Z. J. Haas, ``Analyzing
Split Channel Medium Access Control Schemes with ALOHA
Reservation,"  in {\it Ad-Hoc, Mobile, and Wireless Networks --
ADHOC-NOW '03}, S. Pierre, M. Barbeau, and E. Kranakis, Eds. 2003,
vol. 2865 of Lecture Notes in Computer Science (LNCS), pp.
128-139, Springer-Verlag.

\item W. Du, J. Deng, \underline{Y. S. Han}, and P. K. Varshney,
``A Witness-Based Approach for Data Fusion Assurance in Wireless
Sensor Networks," {\it Proceedings of IEEE 2003 Global
Communications Conference (Globecom2003)}, San Francisco, CA,
December, 2003, pp. 1435-1439.
 \item T.-Y. Wang, \underline{Y. S. Han},
and P. K. Varshney, ``Fault-Tolerant Classification in Multisensor
Networks Using Coding Theory," {\it Proceedings of the 6th
International Conference on Information Fusion (Fusion2003)},
Cairns, Australia, July, 2003, pp. 772-779. {\bf (invited paper)}

\item T.-Y. Wang, P. K. Varshney, and \underline{Y. S. Han},
``Distribution Classification Fusion Using Error Correcting
Codes," {\it Proceedings of the SPIE's Aerosense  conference on
Multisensor, Multisource Information Fusion: Architectures,
Algorithms, and Applications}, Orlando, FL,  April, 2003, pp.
47-57.

\item \underline{Y. S. Han}, P.-N. Chen, and M. Fossorier, ``A
Generalization of the Fano Metric and Its Effect on Sequential
Decoding Using a Stack," {\it Proceedings of the  IEEE
International Symposium on Information Theory}, Lausanne,
Switzerland, June, 2002, p. 134.

\item P.-N. Chen, \underline{Y. S. Han}, C. R. P. Hartmann, and
H.-B. Wu, ``Analysis of Decoding Complexity Using New Variation of
Berry-Esseen Theorem," {\it Proceedings of the  IEEE International
Symposium on Information Theory},  Lausanne, Switzerland, June,
2002, p. 286.

\item C.-K. Lin, P.-N. Chen and \underline{Y. S. Han}, ``A
Low-Complexity Stochastic Codebook Searching Algorithm for
FS1016," {\it Workshop on the 21st Century Digital Life and
Internet Technologies}, Tainan, Taiwan, May, 2001.

\item \underline{Y. S. Han} and P.-N. Chen, ``Asymptotic Covering
Radius of Block Codes ," {\it Proceedings of the International
Symposium on Information theory and Its Applications},  Honolulu,
Hawaii, November, 2000, pp. 521-524.

\item T.-Y. Lee, P.-N. Chen and \underline{Y. S. Han},
"Determination of the Asymptotic Largest Minimum Distance of Block
Codes," {\it Proceedings of the IEEE International Symposium  on
Information Theory}, Sorrento, Italy, June, 2000, p. 227.

\item H.-B. Wu, P.-N. Chen, and \underline{Y. S.
Han},``Investigation of the Maximum-Likelihood Soft-Decision
Sequential Decoding algorithms for convolutional Codes," {\it
Proceedings of the International Symposium on Communications},
Kaohsiung, Taiwan, November, 1999, pp. 82-86.

%\item \underline{Y. S. Han} and P.-N. Chen,
% ``Maximum-Likelihood Soft-Decision Sequential Decoding Algorithms for Convolutional
% Codes,'' invited to present at the recent results session of {\it the 1998 IEEE International
% Symposium  on Information Theory}, Cambridge,
%MA, USA, August, 1998.


\item \underline{Y. S. Han}, ``A Minimum $\rho$-Distance Decoding
Algorithm of Linear Block Codes Based on Voronoi Neighbors, ''
{\it Proceedings of the International Symposium on
Communications}, Hsinchu, Taiwan, December, 1997, pp. 99-103.

\item \underline{Y.  S.  Han},
 ``An Optimal  Gradient Decoding Algorithm  for
Hard-Decision Decoding of Linear Block Codes,'' {\it Proceedings
of the  International Conference on Combinatorics, Information
Theory and Statistics}, Portland, Maine, July, 1997, p. 36. {\bf
(invited speaker)}

\item  \underline{Y.  S.  Han},
 ``A New Treatment of Priority-First Search Maximum-Likelihood
Soft-Decision Decoding for Linear Block Codes,'' {\it Proceedings
of the IEEE International  Symposium  on Information Theory}, Ulm,
Germany, June, 1997, p. 394. {\bf (honored as long presentation)}

\item \underline{Y.  S.  Han},
 ``The Zero-Coverings Algorithm for General Minimum Distance
Decoding Problem,'' {\it Proceedings of the IEEE International
Symposium on Information Theory}, Ulm, Germany, June, 1997, p.
330.

\item \underline{Y.  S.  Han}, ``The Effect of Heuristic
Information on the Soft-Decision Decoding for Linear Block
Codes,"{\it Proceedings of the Seventh IEEE  International
Symposium on Personal, Indoor and Mobile Radio Communications},
Taipei, Taiwan, October, 1996. pp. 309-311.

\item  \underline{Y.  S.  Han},  C.   R.   P.   Hartmann,  C.-T. Chin, and C. K. Mohan,
 ``Efficient Suboptimal Decoding   of Linear   Block  Codes  ,'' {\it Proceedings of the
32nd Allerton Conference on Communication, Control, and
Computing}, University of Illinois, Urbana-Champaign, September,
1994, pp. 93-102. {\bf (invited paper)}

\item  \underline{Y.  S.  Han},  C.   R.   P.   Hartmann,  and  K.   G.  Mehrotra,
 ``Further   Results   on  Decoding   Linear   Block  Codes  Using   a
Generalized  Dijkstra's  Algorithm,'' {\it Proceedings of the 1994
IEEE International  Symposium  on Information Theory}, Trondheim,
Norway, June, 1994, p. 342.

 \item \underline{Y.  S.  Han}, C.  R.  P.
Hartmann, and C-C.  Chen, ``Efficient Maximum-Likelihood Soft-Decision
Decoding of Linear Block Codes Using Algorithm  A*, " {\it Proceedings
of the 1993 IEEE International  Symposium on Information  Theory}, San
Antonio, Texas, January 1993, p. 27. {\bf (honored as long
presentation)}

%\item \underline{Y.  S.  Han}, C.  R.  P.  Hartmann, and K.  G.  Mehrotra, ``Efficient
% Suboptimal
%Soft-Decision Decoding Algorithms of Linear Block Codes Using a
%Generalization of Algorithm A*," Presented at the recent result
%session of the 1993 {\it  IEEE International Symposium on
%Information Theory}, San Antonio, Texas, January 1993.

\item D. L. Tao, \underline{Y.  S.  Han}, and C. R. P. Hartmann, ``New Encoding/Decoding Methods
for Designing Fault-Tolerant Matrix Operations," {\it Proceedings of SPIE, Vol.
1770, Advanced Signal Processing, Algorithms, Architectures, and
Implementations III}, pp. 72-83, July 1992.
\end{enumerate}
\item {\bf\large Technical reports}
\begin{enumerate}
\item \underline{Y.  S.  Han}, and C. R. P. Hartmann, ``Designing Efficient Maximum-Likelihood
Soft-Decision Decoding of Linear Block Codes Using Algorithm A*," Technical
Report SU-CIS-92-10, School of  Computer and Information Science, Syracuse
University, Syracuse, NY, June 1992.

\item \underline{Y.  S.  Han}, C.  R.  P.  Hartmann, and C-C Chen, ``Efficient
Maximum-Likelihood Soft-Decision Decoding of Linear Block Codes Using
Algorithm A*," Technical Report SU-CIS-91-42, School of Computer and
Information Science, Syracuse University, Syracuse, NY, December 1991.
\end{enumerate}
\end{itemize}
\end{llist}
\end{document}
